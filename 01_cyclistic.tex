% Options for packages loaded elsewhere
\PassOptionsToPackage{unicode}{hyperref}
\PassOptionsToPackage{hyphens}{url}
%
\documentclass[
]{article}
\title{01\_\_cyclistic}
\author{Poramed Tripop}
\date{9/30/2021}

\usepackage{amsmath,amssymb}
\usepackage{lmodern}
\usepackage{iftex}
\ifPDFTeX
  \usepackage[T1]{fontenc}
  \usepackage[utf8]{inputenc}
  \usepackage{textcomp} % provide euro and other symbols
\else % if luatex or xetex
  \usepackage{unicode-math}
  \defaultfontfeatures{Scale=MatchLowercase}
  \defaultfontfeatures[\rmfamily]{Ligatures=TeX,Scale=1}
\fi
% Use upquote if available, for straight quotes in verbatim environments
\IfFileExists{upquote.sty}{\usepackage{upquote}}{}
\IfFileExists{microtype.sty}{% use microtype if available
  \usepackage[]{microtype}
  \UseMicrotypeSet[protrusion]{basicmath} % disable protrusion for tt fonts
}{}
\makeatletter
\@ifundefined{KOMAClassName}{% if non-KOMA class
  \IfFileExists{parskip.sty}{%
    \usepackage{parskip}
  }{% else
    \setlength{\parindent}{0pt}
    \setlength{\parskip}{6pt plus 2pt minus 1pt}}
}{% if KOMA class
  \KOMAoptions{parskip=half}}
\makeatother
\usepackage{xcolor}
\IfFileExists{xurl.sty}{\usepackage{xurl}}{} % add URL line breaks if available
\IfFileExists{bookmark.sty}{\usepackage{bookmark}}{\usepackage{hyperref}}
\hypersetup{
  pdftitle={01\_\_cyclistic},
  pdfauthor={Poramed Tripop},
  hidelinks,
  pdfcreator={LaTeX via pandoc}}
\urlstyle{same} % disable monospaced font for URLs
\usepackage[margin=1in]{geometry}
\usepackage{color}
\usepackage{fancyvrb}
\newcommand{\VerbBar}{|}
\newcommand{\VERB}{\Verb[commandchars=\\\{\}]}
\DefineVerbatimEnvironment{Highlighting}{Verbatim}{commandchars=\\\{\}}
% Add ',fontsize=\small' for more characters per line
\usepackage{framed}
\definecolor{shadecolor}{RGB}{248,248,248}
\newenvironment{Shaded}{\begin{snugshade}}{\end{snugshade}}
\newcommand{\AlertTok}[1]{\textcolor[rgb]{0.94,0.16,0.16}{#1}}
\newcommand{\AnnotationTok}[1]{\textcolor[rgb]{0.56,0.35,0.01}{\textbf{\textit{#1}}}}
\newcommand{\AttributeTok}[1]{\textcolor[rgb]{0.77,0.63,0.00}{#1}}
\newcommand{\BaseNTok}[1]{\textcolor[rgb]{0.00,0.00,0.81}{#1}}
\newcommand{\BuiltInTok}[1]{#1}
\newcommand{\CharTok}[1]{\textcolor[rgb]{0.31,0.60,0.02}{#1}}
\newcommand{\CommentTok}[1]{\textcolor[rgb]{0.56,0.35,0.01}{\textit{#1}}}
\newcommand{\CommentVarTok}[1]{\textcolor[rgb]{0.56,0.35,0.01}{\textbf{\textit{#1}}}}
\newcommand{\ConstantTok}[1]{\textcolor[rgb]{0.00,0.00,0.00}{#1}}
\newcommand{\ControlFlowTok}[1]{\textcolor[rgb]{0.13,0.29,0.53}{\textbf{#1}}}
\newcommand{\DataTypeTok}[1]{\textcolor[rgb]{0.13,0.29,0.53}{#1}}
\newcommand{\DecValTok}[1]{\textcolor[rgb]{0.00,0.00,0.81}{#1}}
\newcommand{\DocumentationTok}[1]{\textcolor[rgb]{0.56,0.35,0.01}{\textbf{\textit{#1}}}}
\newcommand{\ErrorTok}[1]{\textcolor[rgb]{0.64,0.00,0.00}{\textbf{#1}}}
\newcommand{\ExtensionTok}[1]{#1}
\newcommand{\FloatTok}[1]{\textcolor[rgb]{0.00,0.00,0.81}{#1}}
\newcommand{\FunctionTok}[1]{\textcolor[rgb]{0.00,0.00,0.00}{#1}}
\newcommand{\ImportTok}[1]{#1}
\newcommand{\InformationTok}[1]{\textcolor[rgb]{0.56,0.35,0.01}{\textbf{\textit{#1}}}}
\newcommand{\KeywordTok}[1]{\textcolor[rgb]{0.13,0.29,0.53}{\textbf{#1}}}
\newcommand{\NormalTok}[1]{#1}
\newcommand{\OperatorTok}[1]{\textcolor[rgb]{0.81,0.36,0.00}{\textbf{#1}}}
\newcommand{\OtherTok}[1]{\textcolor[rgb]{0.56,0.35,0.01}{#1}}
\newcommand{\PreprocessorTok}[1]{\textcolor[rgb]{0.56,0.35,0.01}{\textit{#1}}}
\newcommand{\RegionMarkerTok}[1]{#1}
\newcommand{\SpecialCharTok}[1]{\textcolor[rgb]{0.00,0.00,0.00}{#1}}
\newcommand{\SpecialStringTok}[1]{\textcolor[rgb]{0.31,0.60,0.02}{#1}}
\newcommand{\StringTok}[1]{\textcolor[rgb]{0.31,0.60,0.02}{#1}}
\newcommand{\VariableTok}[1]{\textcolor[rgb]{0.00,0.00,0.00}{#1}}
\newcommand{\VerbatimStringTok}[1]{\textcolor[rgb]{0.31,0.60,0.02}{#1}}
\newcommand{\WarningTok}[1]{\textcolor[rgb]{0.56,0.35,0.01}{\textbf{\textit{#1}}}}
\usepackage{graphicx}
\makeatletter
\def\maxwidth{\ifdim\Gin@nat@width>\linewidth\linewidth\else\Gin@nat@width\fi}
\def\maxheight{\ifdim\Gin@nat@height>\textheight\textheight\else\Gin@nat@height\fi}
\makeatother
% Scale images if necessary, so that they will not overflow the page
% margins by default, and it is still possible to overwrite the defaults
% using explicit options in \includegraphics[width, height, ...]{}
\setkeys{Gin}{width=\maxwidth,height=\maxheight,keepaspectratio}
% Set default figure placement to htbp
\makeatletter
\def\fps@figure{htbp}
\makeatother
\setlength{\emergencystretch}{3em} % prevent overfull lines
\providecommand{\tightlist}{%
  \setlength{\itemsep}{0pt}\setlength{\parskip}{0pt}}
\setcounter{secnumdepth}{-\maxdimen} % remove section numbering
\ifLuaTeX
  \usepackage{selnolig}  % disable illegal ligatures
\fi

\begin{document}
\maketitle

\begin{Shaded}
\begin{Highlighting}[]
\NormalTok{knitr}\SpecialCharTok{::}\NormalTok{opts\_chunk}\SpecialCharTok{$}\FunctionTok{set}\NormalTok{(}\AttributeTok{eval =} \ConstantTok{TRUE}\NormalTok{)}
\end{Highlighting}
\end{Shaded}

\hypertarget{introduction-and-background}{%
\subsubsection{Introduction and
background}\label{introduction-and-background}}

This analysis is based on the \ldots\ldots\ldots{} (found here:
\url{https://artscience.blog/home/divvy-dataviz-case-study}). The
purpose of this script is to consolidate downloaded Divvy data into a
single dataframe and then conduct simple analysis to help answer the key
question: ``In what ways do members and casual riders use Divvy bikes
differently?''

\url{https://divvy-tripdata.s3.amazonaws.com/index.html}

\hypertarget{install-required-packages}{%
\subsubsection{Install required
packages}\label{install-required-packages}}

tidyverse for data import and wrangling lubridate for date functions

\begin{Shaded}
\begin{Highlighting}[]
\CommentTok{\#install.packages("tidyverse")}
\CommentTok{\#install.packages("lubridate")}
\end{Highlighting}
\end{Shaded}

\begin{Shaded}
\begin{Highlighting}[]
\FunctionTok{library}\NormalTok{(tidyverse)}
\end{Highlighting}
\end{Shaded}

\begin{verbatim}
## -- Attaching packages --------------------------------------- tidyverse 1.3.1 --
\end{verbatim}

\begin{verbatim}
## v ggplot2 3.3.5     v purrr   0.3.4
## v tibble  3.1.5     v dplyr   1.0.7
## v tidyr   1.1.4     v stringr 1.4.0
## v readr   2.0.2     v forcats 0.5.1
\end{verbatim}

\begin{verbatim}
## -- Conflicts ------------------------------------------ tidyverse_conflicts() --
## x dplyr::filter() masks stats::filter()
## x dplyr::lag()    masks stats::lag()
\end{verbatim}

\begin{Shaded}
\begin{Highlighting}[]
\FunctionTok{library}\NormalTok{(lubridate)}
\end{Highlighting}
\end{Shaded}

\begin{verbatim}
## 
## Attaching package: 'lubridate'
\end{verbatim}

\begin{verbatim}
## The following objects are masked from 'package:base':
## 
##     date, intersect, setdiff, union
\end{verbatim}

\begin{Shaded}
\begin{Highlighting}[]
\FunctionTok{getwd}\NormalTok{()}
\end{Highlighting}
\end{Shaded}

\begin{verbatim}
## [1] "C:/Users/Action/Documents/RStudio/repos/01__cyclistic"
\end{verbatim}

\begin{Shaded}
\begin{Highlighting}[]
\CommentTok{\#setwd("C:/Users/Action/Documents/RStudio/GGDA/Project/Cyclistic") \#sets your working directory to simplify calls to data}
\end{Highlighting}
\end{Shaded}

\hypertarget{step-1-collect-data}{%
\subsubsection{STEP 1: COLLECT DATA}\label{step-1-collect-data}}

\hypertarget{upload-cyclistic-datasets-csv-files-here}{%
\paragraph{Upload Cyclistic datasets (csv files)
here}\label{upload-cyclistic-datasets-csv-files-here}}

\url{https://divvy-tripdata.s3.amazonaws.com/index.html}

Divvy\_Trips\_2019\_Q2 is not csv file. So, open it with Excel and save
it as csv file before upload to the project.

\begin{Shaded}
\begin{Highlighting}[]
\NormalTok{q2\_2019 }\OtherTok{\textless{}{-}} \FunctionTok{read\_csv}\NormalTok{(}\StringTok{"Divvy\_Trips\_2019\_Q2.csv"}\NormalTok{)}
\end{Highlighting}
\end{Shaded}

\begin{verbatim}
## Rows: 1048575 Columns: 12
\end{verbatim}

\begin{verbatim}
## -- Column specification --------------------------------------------------------
## Delimiter: ","
## chr (6): 01 - Rental Details Local Start Time, 01 - Rental Details Local End...
## dbl (5): 01 - Rental Details Rental ID, 01 - Rental Details Bike ID, 03 - Re...
\end{verbatim}

\begin{verbatim}
## 
## i Use `spec()` to retrieve the full column specification for this data.
## i Specify the column types or set `show_col_types = FALSE` to quiet this message.
\end{verbatim}

\begin{Shaded}
\begin{Highlighting}[]
\NormalTok{q3\_2019 }\OtherTok{\textless{}{-}} \FunctionTok{read\_csv}\NormalTok{(}\StringTok{"Divvy\_Trips\_2019\_Q3.csv"}\NormalTok{)}
\end{Highlighting}
\end{Shaded}

\begin{verbatim}
## Rows: 1640718 Columns: 12
\end{verbatim}

\begin{verbatim}
## -- Column specification --------------------------------------------------------
## Delimiter: ","
## chr  (4): from_station_name, to_station_name, usertype, gender
## dbl  (5): trip_id, bikeid, from_station_id, to_station_id, birthyear
## dttm (2): start_time, end_time
\end{verbatim}

\begin{verbatim}
## 
## i Use `spec()` to retrieve the full column specification for this data.
## i Specify the column types or set `show_col_types = FALSE` to quiet this message.
\end{verbatim}

\begin{Shaded}
\begin{Highlighting}[]
\NormalTok{q4\_2019 }\OtherTok{\textless{}{-}} \FunctionTok{read\_csv}\NormalTok{(}\StringTok{"Divvy\_Trips\_2019\_Q4.csv"}\NormalTok{)}
\end{Highlighting}
\end{Shaded}

\begin{verbatim}
## Rows: 704054 Columns: 12
\end{verbatim}

\begin{verbatim}
## -- Column specification --------------------------------------------------------
## Delimiter: ","
## chr  (4): from_station_name, to_station_name, usertype, gender
## dbl  (5): trip_id, bikeid, from_station_id, to_station_id, birthyear
## dttm (2): start_time, end_time
\end{verbatim}

\begin{verbatim}
## 
## i Use `spec()` to retrieve the full column specification for this data.
## i Specify the column types or set `show_col_types = FALSE` to quiet this message.
\end{verbatim}

\begin{Shaded}
\begin{Highlighting}[]
\NormalTok{q1\_2020 }\OtherTok{\textless{}{-}} \FunctionTok{read\_csv}\NormalTok{(}\StringTok{"Divvy\_Trips\_2020\_Q1.csv"}\NormalTok{)}
\end{Highlighting}
\end{Shaded}

\begin{verbatim}
## Rows: 426887 Columns: 13
\end{verbatim}

\begin{verbatim}
## -- Column specification --------------------------------------------------------
## Delimiter: ","
## chr  (5): ride_id, rideable_type, start_station_name, end_station_name, memb...
## dbl  (6): start_station_id, end_station_id, start_lat, start_lng, end_lat, e...
## dttm (2): started_at, ended_at
\end{verbatim}

\begin{verbatim}
## 
## i Use `spec()` to retrieve the full column specification for this data.
## i Specify the column types or set `show_col_types = FALSE` to quiet this message.
\end{verbatim}

\hypertarget{step-2-wrangle-data-and-combine-into-a-single-file}{%
\subsubsection{STEP 2: WRANGLE DATA AND COMBINE INTO A SINGLE
FILE}\label{step-2-wrangle-data-and-combine-into-a-single-file}}

\hypertarget{compare-column-names-each-of-the-files}{%
\paragraph{Compare column names each of the
files}\label{compare-column-names-each-of-the-files}}

\hypertarget{while-the-names-dont-have-to-be-in-the-same-order-they-do-need-to-match-perfectly-before-we-can-use-a-command-to-join-them-into-one-file}{%
\paragraph{While the names don't have to be in the same order, they DO
need to match perfectly before we can use a command to join them into
one
file}\label{while-the-names-dont-have-to-be-in-the-same-order-they-do-need-to-match-perfectly-before-we-can-use-a-command-to-join-them-into-one-file}}

\begin{Shaded}
\begin{Highlighting}[]
\FunctionTok{colnames}\NormalTok{(q2\_2019)}
\end{Highlighting}
\end{Shaded}

\begin{verbatim}
##  [1] "01 - Rental Details Rental ID"                   
##  [2] "01 - Rental Details Local Start Time"            
##  [3] "01 - Rental Details Local End Time"              
##  [4] "01 - Rental Details Bike ID"                     
##  [5] "01 - Rental Details Duration In Seconds Uncapped"
##  [6] "03 - Rental Start Station ID"                    
##  [7] "03 - Rental Start Station Name"                  
##  [8] "02 - Rental End Station ID"                      
##  [9] "02 - Rental End Station Name"                    
## [10] "User Type"                                       
## [11] "Member Gender"                                   
## [12] "05 - Member Details Member Birthday Year"
\end{verbatim}

\begin{Shaded}
\begin{Highlighting}[]
\FunctionTok{colnames}\NormalTok{(q3\_2019)}
\end{Highlighting}
\end{Shaded}

\begin{verbatim}
##  [1] "trip_id"           "start_time"        "end_time"         
##  [4] "bikeid"            "tripduration"      "from_station_id"  
##  [7] "from_station_name" "to_station_id"     "to_station_name"  
## [10] "usertype"          "gender"            "birthyear"
\end{verbatim}

\begin{Shaded}
\begin{Highlighting}[]
\FunctionTok{colnames}\NormalTok{(q4\_2019)}
\end{Highlighting}
\end{Shaded}

\begin{verbatim}
##  [1] "trip_id"           "start_time"        "end_time"         
##  [4] "bikeid"            "tripduration"      "from_station_id"  
##  [7] "from_station_name" "to_station_id"     "to_station_name"  
## [10] "usertype"          "gender"            "birthyear"
\end{verbatim}

\begin{Shaded}
\begin{Highlighting}[]
\FunctionTok{colnames}\NormalTok{(q1\_2020)}
\end{Highlighting}
\end{Shaded}

\begin{verbatim}
##  [1] "ride_id"            "rideable_type"      "started_at"        
##  [4] "ended_at"           "start_station_name" "start_station_id"  
##  [7] "end_station_name"   "end_station_id"     "start_lat"         
## [10] "start_lng"          "end_lat"            "end_lng"           
## [13] "member_casual"
\end{verbatim}

\hypertarget{rename-columns-to-make-them-consisent-with-q1_2020-as-this-will-be-the-supposed-going-forward-table-design-for-cyclistic}{%
\paragraph{Rename columns to make them consisent with q1\_2020 (as this
will be the supposed going-forward table design for
Cyclistic)}\label{rename-columns-to-make-them-consisent-with-q1_2020-as-this-will-be-the-supposed-going-forward-table-design-for-cyclistic}}

\begin{Shaded}
\begin{Highlighting}[]
\NormalTok{q2\_2019 }\OtherTok{\textless{}{-}} \FunctionTok{rename}\NormalTok{(q2\_2019}
\NormalTok{                   ,}\AttributeTok{ride\_id =} \StringTok{"01 {-} Rental Details Rental ID"}
\NormalTok{                   ,}\AttributeTok{rideable\_type =} \StringTok{"01 {-} Rental Details Bike ID"} 
\NormalTok{                   ,}\AttributeTok{started\_at =} \StringTok{"01 {-} Rental Details Local Start Time"}  
\NormalTok{                   ,}\AttributeTok{ended\_at =} \StringTok{"01 {-} Rental Details Local End Time"}  
\NormalTok{                   ,}\AttributeTok{start\_station\_name =} \StringTok{"03 {-} Rental Start Station Name"} 
\NormalTok{                   ,}\AttributeTok{start\_station\_id =} \StringTok{"03 {-} Rental Start Station ID"}
\NormalTok{                   ,}\AttributeTok{end\_station\_name =} \StringTok{"02 {-} Rental End Station Name"} 
\NormalTok{                   ,}\AttributeTok{end\_station\_id =} \StringTok{"02 {-} Rental End Station ID"}
\NormalTok{                   ,}\AttributeTok{member\_casual =} \StringTok{"User Type"}\NormalTok{)}

\NormalTok{q3\_2019 }\OtherTok{\textless{}{-}} \FunctionTok{rename}\NormalTok{(q3\_2019}
\NormalTok{                   ,}\AttributeTok{ride\_id =}\NormalTok{ trip\_id}
\NormalTok{                   ,}\AttributeTok{rideable\_type =}\NormalTok{ bikeid }
\NormalTok{                   ,}\AttributeTok{started\_at =}\NormalTok{ start\_time  }
\NormalTok{                   ,}\AttributeTok{ended\_at =}\NormalTok{ end\_time  }
\NormalTok{                   ,}\AttributeTok{start\_station\_name =}\NormalTok{ from\_station\_name }
\NormalTok{                   ,}\AttributeTok{start\_station\_id =}\NormalTok{ from\_station\_id }
\NormalTok{                   ,}\AttributeTok{end\_station\_name =}\NormalTok{ to\_station\_name }
\NormalTok{                   ,}\AttributeTok{end\_station\_id =}\NormalTok{ to\_station\_id }
\NormalTok{                   ,}\AttributeTok{member\_casual =}\NormalTok{ usertype)}

\NormalTok{q4\_2019 }\OtherTok{\textless{}{-}} \FunctionTok{rename}\NormalTok{(q4\_2019}
\NormalTok{                   ,}\AttributeTok{ride\_id =}\NormalTok{ trip\_id}
\NormalTok{                   ,}\AttributeTok{rideable\_type =}\NormalTok{ bikeid }
\NormalTok{                   ,}\AttributeTok{started\_at =}\NormalTok{ start\_time  }
\NormalTok{                   ,}\AttributeTok{ended\_at =}\NormalTok{ end\_time  }
\NormalTok{                   ,}\AttributeTok{start\_station\_name =}\NormalTok{ from\_station\_name }
\NormalTok{                   ,}\AttributeTok{start\_station\_id =}\NormalTok{ from\_station\_id }
\NormalTok{                   ,}\AttributeTok{end\_station\_name =}\NormalTok{ to\_station\_name }
\NormalTok{                   ,}\AttributeTok{end\_station\_id =}\NormalTok{ to\_station\_id }
\NormalTok{                   ,}\AttributeTok{member\_casual =}\NormalTok{ usertype)}
\end{Highlighting}
\end{Shaded}

\hypertarget{inspect-the-dataframes-and-look-for-incongruencies}{%
\paragraph{Inspect the dataframes and look for
incongruencies}\label{inspect-the-dataframes-and-look-for-incongruencies}}

\begin{Shaded}
\begin{Highlighting}[]
\FunctionTok{str}\NormalTok{(q2\_2019)}
\end{Highlighting}
\end{Shaded}

\begin{verbatim}
## spec_tbl_df [1,048,575 x 12] (S3: spec_tbl_df/tbl_df/tbl/data.frame)
##  $ ride_id                                         : num [1:1048575] 22178529 22178530 22178531 22178532 22178533 ...
##  $ started_at                                      : chr [1:1048575] "01-04-19 0:02" "01-04-19 0:03" "01-04-19 0:11" "01-04-19 0:13" ...
##  $ ended_at                                        : chr [1:1048575] "01-04-19 0:09" "01-04-19 0:20" "01-04-19 0:15" "01-04-19 0:18" ...
##  $ rideable_type                                   : num [1:1048575] 6251 6226 5649 4151 3270 ...
##  $ 01 - Rental Details Duration In Seconds Uncapped: num [1:1048575] 446 1048 252 357 1007 ...
##  $ start_station_id                                : num [1:1048575] 81 317 283 26 202 420 503 260 211 211 ...
##  $ start_station_name                              : chr [1:1048575] "Daley Center Plaza" "Wood St & Taylor St" "LaSalle St & Jackson Blvd" "McClurg Ct & Illinois St" ...
##  $ end_station_id                                  : num [1:1048575] 56 59 174 133 129 426 500 499 211 211 ...
##  $ end_station_name                                : chr [1:1048575] "Desplaines St & Kinzie St" "Wabash Ave & Roosevelt Rd" "Canal St & Madison St" "Kingsbury St & Kinzie St" ...
##  $ member_casual                                   : chr [1:1048575] "Subscriber" "Subscriber" "Subscriber" "Subscriber" ...
##  $ Member Gender                                   : chr [1:1048575] "Male" "Female" "Male" "Male" ...
##  $ 05 - Member Details Member Birthday Year        : num [1:1048575] 1975 1984 1990 1993 1992 ...
##  - attr(*, "spec")=
##   .. cols(
##   ..   `01 - Rental Details Rental ID` = col_double(),
##   ..   `01 - Rental Details Local Start Time` = col_character(),
##   ..   `01 - Rental Details Local End Time` = col_character(),
##   ..   `01 - Rental Details Bike ID` = col_double(),
##   ..   `01 - Rental Details Duration In Seconds Uncapped` = col_number(),
##   ..   `03 - Rental Start Station ID` = col_double(),
##   ..   `03 - Rental Start Station Name` = col_character(),
##   ..   `02 - Rental End Station ID` = col_double(),
##   ..   `02 - Rental End Station Name` = col_character(),
##   ..   `User Type` = col_character(),
##   ..   `Member Gender` = col_character(),
##   ..   `05 - Member Details Member Birthday Year` = col_double()
##   .. )
##  - attr(*, "problems")=<externalptr>
\end{verbatim}

\begin{Shaded}
\begin{Highlighting}[]
\FunctionTok{str}\NormalTok{(q3\_2019)}
\end{Highlighting}
\end{Shaded}

\begin{verbatim}
## spec_tbl_df [1,640,718 x 12] (S3: spec_tbl_df/tbl_df/tbl/data.frame)
##  $ ride_id           : num [1:1640718] 23479388 23479389 23479390 23479391 23479392 ...
##  $ started_at        : POSIXct[1:1640718], format: "2019-07-01 00:00:27" "2019-07-01 00:01:16" ...
##  $ ended_at          : POSIXct[1:1640718], format: "2019-07-01 00:20:41" "2019-07-01 00:18:44" ...
##  $ rideable_type     : num [1:1640718] 3591 5353 6180 5540 6014 ...
##  $ tripduration      : num [1:1640718] 1214 1048 1554 1503 1213 ...
##  $ start_station_id  : num [1:1640718] 117 381 313 313 168 300 168 313 43 43 ...
##  $ start_station_name: chr [1:1640718] "Wilton Ave & Belmont Ave" "Western Ave & Monroe St" "Lakeview Ave & Fullerton Pkwy" "Lakeview Ave & Fullerton Pkwy" ...
##  $ end_station_id    : num [1:1640718] 497 203 144 144 62 232 62 144 195 195 ...
##  $ end_station_name  : chr [1:1640718] "Kimball Ave & Belmont Ave" "Western Ave & 21st St" "Larrabee St & Webster Ave" "Larrabee St & Webster Ave" ...
##  $ member_casual     : chr [1:1640718] "Subscriber" "Customer" "Customer" "Customer" ...
##  $ gender            : chr [1:1640718] "Male" NA NA NA ...
##  $ birthyear         : num [1:1640718] 1992 NA NA NA NA ...
##  - attr(*, "spec")=
##   .. cols(
##   ..   trip_id = col_double(),
##   ..   start_time = col_datetime(format = ""),
##   ..   end_time = col_datetime(format = ""),
##   ..   bikeid = col_double(),
##   ..   tripduration = col_number(),
##   ..   from_station_id = col_double(),
##   ..   from_station_name = col_character(),
##   ..   to_station_id = col_double(),
##   ..   to_station_name = col_character(),
##   ..   usertype = col_character(),
##   ..   gender = col_character(),
##   ..   birthyear = col_double()
##   .. )
##  - attr(*, "problems")=<externalptr>
\end{verbatim}

\begin{Shaded}
\begin{Highlighting}[]
\FunctionTok{str}\NormalTok{(q4\_2019)}
\end{Highlighting}
\end{Shaded}

\begin{verbatim}
## spec_tbl_df [704,054 x 12] (S3: spec_tbl_df/tbl_df/tbl/data.frame)
##  $ ride_id           : num [1:704054] 25223640 25223641 25223642 25223643 25223644 ...
##  $ started_at        : POSIXct[1:704054], format: "2019-10-01 00:01:39" "2019-10-01 00:02:16" ...
##  $ ended_at          : POSIXct[1:704054], format: "2019-10-01 00:17:20" "2019-10-01 00:06:34" ...
##  $ rideable_type     : num [1:704054] 2215 6328 3003 3275 5294 ...
##  $ tripduration      : num [1:704054] 940 258 850 2350 1867 ...
##  $ start_station_id  : num [1:704054] 20 19 84 313 210 156 84 156 156 336 ...
##  $ start_station_name: chr [1:704054] "Sheffield Ave & Kingsbury St" "Throop (Loomis) St & Taylor St" "Milwaukee Ave & Grand Ave" "Lakeview Ave & Fullerton Pkwy" ...
##  $ end_station_id    : num [1:704054] 309 241 199 290 382 226 142 463 463 336 ...
##  $ end_station_name  : chr [1:704054] "Leavitt St & Armitage Ave" "Morgan St & Polk St" "Wabash Ave & Grand Ave" "Kedzie Ave & Palmer Ct" ...
##  $ member_casual     : chr [1:704054] "Subscriber" "Subscriber" "Subscriber" "Subscriber" ...
##  $ gender            : chr [1:704054] "Male" "Male" "Female" "Male" ...
##  $ birthyear         : num [1:704054] 1987 1998 1991 1990 1987 ...
##  - attr(*, "spec")=
##   .. cols(
##   ..   trip_id = col_double(),
##   ..   start_time = col_datetime(format = ""),
##   ..   end_time = col_datetime(format = ""),
##   ..   bikeid = col_double(),
##   ..   tripduration = col_number(),
##   ..   from_station_id = col_double(),
##   ..   from_station_name = col_character(),
##   ..   to_station_id = col_double(),
##   ..   to_station_name = col_character(),
##   ..   usertype = col_character(),
##   ..   gender = col_character(),
##   ..   birthyear = col_double()
##   .. )
##  - attr(*, "problems")=<externalptr>
\end{verbatim}

\begin{Shaded}
\begin{Highlighting}[]
\FunctionTok{str}\NormalTok{(q1\_2020)}
\end{Highlighting}
\end{Shaded}

\begin{verbatim}
## spec_tbl_df [426,887 x 13] (S3: spec_tbl_df/tbl_df/tbl/data.frame)
##  $ ride_id           : chr [1:426887] "EACB19130B0CDA4A" "8FED874C809DC021" "789F3C21E472CA96" "C9A388DAC6ABF313" ...
##  $ rideable_type     : chr [1:426887] "docked_bike" "docked_bike" "docked_bike" "docked_bike" ...
##  $ started_at        : POSIXct[1:426887], format: "2020-01-21 20:06:59" "2020-01-30 14:22:39" ...
##  $ ended_at          : POSIXct[1:426887], format: "2020-01-21 20:14:30" "2020-01-30 14:26:22" ...
##  $ start_station_name: chr [1:426887] "Western Ave & Leland Ave" "Clark St & Montrose Ave" "Broadway & Belmont Ave" "Clark St & Randolph St" ...
##  $ start_station_id  : num [1:426887] 239 234 296 51 66 212 96 96 212 38 ...
##  $ end_station_name  : chr [1:426887] "Clark St & Leland Ave" "Southport Ave & Irving Park Rd" "Wilton Ave & Belmont Ave" "Fairbanks Ct & Grand Ave" ...
##  $ end_station_id    : num [1:426887] 326 318 117 24 212 96 212 212 96 100 ...
##  $ start_lat         : num [1:426887] 42 42 41.9 41.9 41.9 ...
##  $ start_lng         : num [1:426887] -87.7 -87.7 -87.6 -87.6 -87.6 ...
##  $ end_lat           : num [1:426887] 42 42 41.9 41.9 41.9 ...
##  $ end_lng           : num [1:426887] -87.7 -87.7 -87.7 -87.6 -87.6 ...
##  $ member_casual     : chr [1:426887] "member" "member" "member" "member" ...
##  - attr(*, "spec")=
##   .. cols(
##   ..   ride_id = col_character(),
##   ..   rideable_type = col_character(),
##   ..   started_at = col_datetime(format = ""),
##   ..   ended_at = col_datetime(format = ""),
##   ..   start_station_name = col_character(),
##   ..   start_station_id = col_double(),
##   ..   end_station_name = col_character(),
##   ..   end_station_id = col_double(),
##   ..   start_lat = col_double(),
##   ..   start_lng = col_double(),
##   ..   end_lat = col_double(),
##   ..   end_lng = col_double(),
##   ..   member_casual = col_character()
##   .. )
##  - attr(*, "problems")=<externalptr>
\end{verbatim}

\hypertarget{convert-ride_id-and-rideable_type-to-character-so-that-they-can-stack-correctly}{%
\paragraph{Convert ride\_id and rideable\_type to character so that they
can stack
correctly}\label{convert-ride_id-and-rideable_type-to-character-so-that-they-can-stack-correctly}}

\begin{Shaded}
\begin{Highlighting}[]
\NormalTok{q2\_2019 }\OtherTok{\textless{}{-}}  \FunctionTok{mutate}\NormalTok{(q2\_2019, }\AttributeTok{ride\_id =} \FunctionTok{as.character}\NormalTok{(ride\_id)}
\NormalTok{                   ,}\AttributeTok{rideable\_type =} \FunctionTok{as.character}\NormalTok{(rideable\_type))}

\NormalTok{q3\_2019 }\OtherTok{\textless{}{-}}  \FunctionTok{mutate}\NormalTok{(q3\_2019, }\AttributeTok{ride\_id =} \FunctionTok{as.character}\NormalTok{(ride\_id)}
\NormalTok{                   ,}\AttributeTok{rideable\_type =} \FunctionTok{as.character}\NormalTok{(rideable\_type))}

\NormalTok{q4\_2019 }\OtherTok{\textless{}{-}}  \FunctionTok{mutate}\NormalTok{(q4\_2019, }\AttributeTok{ride\_id =} \FunctionTok{as.character}\NormalTok{(ride\_id)}
\NormalTok{                   ,}\AttributeTok{rideable\_type =} \FunctionTok{as.character}\NormalTok{(rideable\_type)) }
\end{Highlighting}
\end{Shaded}

\hypertarget{convert-started_at-and-ended_at-in-q2_2019-from-character-to-datetime}{%
\paragraph{convert started\_at and ended\_at in q2\_2019 from character
to
datetime}\label{convert-started_at-and-ended_at-in-q2_2019-from-character-to-datetime}}

\begin{Shaded}
\begin{Highlighting}[]
\NormalTok{q2\_2019 }\OtherTok{\textless{}{-}}  \FunctionTok{mutate}\NormalTok{(q2\_2019, }\AttributeTok{started\_at =} \FunctionTok{dmy\_hm}\NormalTok{(started\_at, }\AttributeTok{tz =} \StringTok{"UTC"}\NormalTok{)}
\NormalTok{                   ,}\AttributeTok{ended\_at =} \FunctionTok{dmy\_hm}\NormalTok{(ended\_at, }\AttributeTok{tz =} \StringTok{"UTC"}\NormalTok{))}
\end{Highlighting}
\end{Shaded}

\hypertarget{stack-individual-quarters-data-frames-into-one-big-data-frame}{%
\paragraph{Stack individual quarter's data frames into one big data
frame}\label{stack-individual-quarters-data-frames-into-one-big-data-frame}}

\begin{Shaded}
\begin{Highlighting}[]
\NormalTok{all\_trips }\OtherTok{\textless{}{-}} \FunctionTok{bind\_rows}\NormalTok{(q2\_2019, q3\_2019, q4\_2019, q1\_2020)}
\end{Highlighting}
\end{Shaded}

\hypertarget{remove-lat-long-birthyear-and-gender-fields-as-this-data-was-dropped-beginning-in-2020}{%
\paragraph{Remove lat, long, birthyear, and gender fields as this data
was dropped beginning in
2020}\label{remove-lat-long-birthyear-and-gender-fields-as-this-data-was-dropped-beginning-in-2020}}

\begin{Shaded}
\begin{Highlighting}[]
\NormalTok{all\_trips }\OtherTok{\textless{}{-}}\NormalTok{ all\_trips }\SpecialCharTok{\%\textgreater{}\%}  
  \FunctionTok{select}\NormalTok{(}\SpecialCharTok{{-}}\FunctionTok{c}\NormalTok{(start\_lat, start\_lng, end\_lat, end\_lng, }
\NormalTok{            birthyear, gender, }
            \StringTok{"01 {-} Rental Details Duration In Seconds Uncapped"}\NormalTok{, }
            \StringTok{"05 {-} Member Details Member Birthday Year"}\NormalTok{, }
            \StringTok{"Member Gender"}\NormalTok{, }\StringTok{"tripduration"}\NormalTok{))}
\end{Highlighting}
\end{Shaded}

\hypertarget{step-3-clean-up-and-add-data-to-prepare-for-analysis}{%
\subsubsection{STEP 3: CLEAN UP AND ADD DATA TO PREPARE FOR
ANALYSIS}\label{step-3-clean-up-and-add-data-to-prepare-for-analysis}}

\hypertarget{inspect-the-new-table-that-has-been-created}{%
\paragraph{Inspect the new table that has been
created}\label{inspect-the-new-table-that-has-been-created}}

\begin{Shaded}
\begin{Highlighting}[]
\FunctionTok{colnames}\NormalTok{(all\_trips)}
\end{Highlighting}
\end{Shaded}

\begin{verbatim}
## [1] "ride_id"            "started_at"         "ended_at"          
## [4] "rideable_type"      "start_station_id"   "start_station_name"
## [7] "end_station_id"     "end_station_name"   "member_casual"
\end{verbatim}

\begin{Shaded}
\begin{Highlighting}[]
\FunctionTok{nrow}\NormalTok{(all\_trips)}
\end{Highlighting}
\end{Shaded}

\begin{verbatim}
## [1] 3820234
\end{verbatim}

\begin{Shaded}
\begin{Highlighting}[]
\FunctionTok{dim}\NormalTok{(all\_trips)}
\end{Highlighting}
\end{Shaded}

\begin{verbatim}
## [1] 3820234       9
\end{verbatim}

\begin{Shaded}
\begin{Highlighting}[]
\FunctionTok{head}\NormalTok{(all\_trips)}
\end{Highlighting}
\end{Shaded}

\begin{verbatim}
## # A tibble: 6 x 9
##   ride_id  started_at          ended_at            rideable_type start_station_id
##   <chr>    <dttm>              <dttm>              <chr>                    <dbl>
## 1 22178529 2019-04-01 00:02:00 2019-04-01 00:09:00 6251                        81
## 2 22178530 2019-04-01 00:03:00 2019-04-01 00:20:00 6226                       317
## 3 22178531 2019-04-01 00:11:00 2019-04-01 00:15:00 5649                       283
## 4 22178532 2019-04-01 00:13:00 2019-04-01 00:18:00 4151                        26
## 5 22178533 2019-04-01 00:19:00 2019-04-01 00:36:00 3270                       202
## 6 22178534 2019-04-01 00:19:00 2019-04-01 00:23:00 3123                       420
## # ... with 4 more variables: start_station_name <chr>, end_station_id <dbl>,
## #   end_station_name <chr>, member_casual <chr>
\end{verbatim}

\begin{Shaded}
\begin{Highlighting}[]
\FunctionTok{tail}\NormalTok{(all\_trips)}
\end{Highlighting}
\end{Shaded}

\begin{verbatim}
## # A tibble: 6 x 9
##   ride_id          started_at          ended_at            rideable_type start_station_id
##   <chr>            <dttm>              <dttm>              <chr>                    <dbl>
## 1 6F4D221BDDFD943F 2020-03-10 10:40:27 2020-03-10 10:40:29 docked_bike                675
## 2 ADDAA33CEBCAE733 2020-03-10 10:40:06 2020-03-10 10:40:07 docked_bike                675
## 3 82B10FA3994BC66A 2020-03-07 15:25:55 2020-03-07 16:14:03 docked_bike                161
## 4 AA0D5AAA0B59C8AA 2020-03-01 13:12:38 2020-03-01 13:38:29 docked_bike                141
## 5 3296360A7BC20FB8 2020-03-07 18:02:45 2020-03-07 18:13:18 docked_bike                672
## 6 064EC7698E4FF9B3 2020-03-08 13:03:57 2020-03-08 13:32:27 docked_bike                110
## # ... with 4 more variables: start_station_name <chr>, end_station_id <dbl>,
## #   end_station_name <chr>, member_casual <chr>
\end{verbatim}

\begin{Shaded}
\begin{Highlighting}[]
\FunctionTok{str}\NormalTok{(all\_trips)}
\end{Highlighting}
\end{Shaded}

\begin{verbatim}
## tibble [3,820,234 x 9] (S3: tbl_df/tbl/data.frame)
##  $ ride_id           : chr [1:3820234] "22178529" "22178530" "22178531" "22178532" ...
##  $ started_at        : POSIXct[1:3820234], format: "2019-04-01 00:02:00" "2019-04-01 00:03:00" ...
##  $ ended_at          : POSIXct[1:3820234], format: "2019-04-01 00:09:00" "2019-04-01 00:20:00" ...
##  $ rideable_type     : chr [1:3820234] "6251" "6226" "5649" "4151" ...
##  $ start_station_id  : num [1:3820234] 81 317 283 26 202 420 503 260 211 211 ...
##  $ start_station_name: chr [1:3820234] "Daley Center Plaza" "Wood St & Taylor St" "LaSalle St & Jackson Blvd" "McClurg Ct & Illinois St" ...
##  $ end_station_id    : num [1:3820234] 56 59 174 133 129 426 500 499 211 211 ...
##  $ end_station_name  : chr [1:3820234] "Desplaines St & Kinzie St" "Wabash Ave & Roosevelt Rd" "Canal St & Madison St" "Kingsbury St & Kinzie St" ...
##  $ member_casual     : chr [1:3820234] "Subscriber" "Subscriber" "Subscriber" "Subscriber" ...
\end{verbatim}

\begin{Shaded}
\begin{Highlighting}[]
\FunctionTok{summary}\NormalTok{(all\_trips)}
\end{Highlighting}
\end{Shaded}

\begin{verbatim}
##    ride_id            started_at                     ended_at                  
##  Length:3820234     Min.   :2019-04-01 00:02:00   Min.   :2019-04-01 00:09:00  
##  Class :character   1st Qu.:2019-06-22 12:40:00   1st Qu.:2019-06-22 13:09:00  
##  Mode  :character   Median :2019-08-16 08:38:02   Median :2019-08-16 08:55:10  
##                     Mean   :2019-08-26 22:25:54   Mean   :2019-08-26 22:50:30  
##                     3rd Qu.:2019-10-13 17:26:07   3rd Qu.:2019-10-13 17:51:07  
##                     Max.   :2020-03-31 23:51:34   Max.   :2020-05-19 20:10:34  
##                                                                                
##  rideable_type      start_station_id start_station_name end_station_id 
##  Length:3820234     Min.   :  1.0    Length:3820234     Min.   :  1.0  
##  Class :character   1st Qu.: 77.0    Class :character   1st Qu.: 77.0  
##  Mode  :character   Median :174.0    Mode  :character   Median :174.0  
##                     Mean   :202.9                       Mean   :203.7  
##                     3rd Qu.:290.0                       3rd Qu.:291.0  
##                     Max.   :675.0                       Max.   :675.0  
##                                                         NA's   :1      
##  end_station_name   member_casual     
##  Length:3820234     Length:3820234    
##  Class :character   Class :character  
##  Mode  :character   Mode  :character  
##                                       
##                                       
##                                       
## 
\end{verbatim}

\hypertarget{there-are-a-few-problems-we-will-need-to-fix}{%
\paragraph{There are a few problems we will need to
fix:}\label{there-are-a-few-problems-we-will-need-to-fix}}

\hypertarget{in-the-member_casual-column-there-are-two-names-for-members-member-and-subscriber-and-two-names-for-casual-riders-customer-and-casual.-we-will-need-to-consolidate-that-from-four-to-two-labels.}{%
\paragraph{(1) In the ``member\_casual'' column, there are two names for
members (``member'' and ``Subscriber'') and two names for casual riders
(``Customer'' and ``casual''). We will need to consolidate that from
four to two
labels.}\label{in-the-member_casual-column-there-are-two-names-for-members-member-and-subscriber-and-two-names-for-casual-riders-customer-and-casual.-we-will-need-to-consolidate-that-from-four-to-two-labels.}}

In the ``member\_casual'' column, replace ``Subscriber'' with ``member''
and ``Customer'' with ``casual''

Before 2020, Divvy used different labels for these two types of riders.
We will make our dataframe consistent with their current nomenclature

N.B.: ``Level'' is a special property of a column that is retained even
if a subset does not contain any values from a specific level

\hypertarget{begin-by-seeing-how-many-observations-fall-under-each-usertype}{%
\subparagraph{Begin by seeing how many observations fall under each
usertype}\label{begin-by-seeing-how-many-observations-fall-under-each-usertype}}

\begin{Shaded}
\begin{Highlighting}[]
\FunctionTok{table}\NormalTok{(all\_trips}\SpecialCharTok{$}\NormalTok{member\_casual)}
\end{Highlighting}
\end{Shaded}

\begin{verbatim}
## 
##     casual   Customer     member Subscriber 
##      48480     833822     378407    2559525
\end{verbatim}

\hypertarget{reassign-to-the-desired-values-we-will-go-with-the-current-2020-labels}{%
\subparagraph{Reassign to the desired values (we will go with the
current 2020
labels)}\label{reassign-to-the-desired-values-we-will-go-with-the-current-2020-labels}}

\begin{Shaded}
\begin{Highlighting}[]
\NormalTok{all\_trips }\OtherTok{\textless{}{-}}\NormalTok{  all\_trips }\SpecialCharTok{\%\textgreater{}\%} 
  \FunctionTok{mutate}\NormalTok{(}\AttributeTok{member\_casual =} \FunctionTok{recode}\NormalTok{(member\_casual}
\NormalTok{                           ,}\StringTok{"Subscriber"} \OtherTok{=} \StringTok{"member"}
\NormalTok{                           ,}\StringTok{"Customer"} \OtherTok{=} \StringTok{"casual"}\NormalTok{))}
\end{Highlighting}
\end{Shaded}

\hypertarget{check-to-make-sure-the-proper-number-of-observations-were-reassigned}{%
\subparagraph{Check to make sure the proper number of observations were
reassigned}\label{check-to-make-sure-the-proper-number-of-observations-were-reassigned}}

\begin{Shaded}
\begin{Highlighting}[]
\FunctionTok{table}\NormalTok{(all\_trips}\SpecialCharTok{$}\NormalTok{member\_casual)}
\end{Highlighting}
\end{Shaded}

\begin{verbatim}
## 
##  casual  member 
##  882302 2937932
\end{verbatim}

\hypertarget{the-data-can-only-be-aggregated-at-the-ride-level-which-is-too-granular.-we-will-want-to-add-some-additional-columns-of-data-such-as-day-month-year-that-provide-additional-opportunities-to-aggregate-the-data.}{%
\paragraph{(2) The data can only be aggregated at the ride-level, which
is too granular. We will want to add some additional columns of data --
such as day, month, year -- that provide additional opportunities to
aggregate the
data.}\label{the-data-can-only-be-aggregated-at-the-ride-level-which-is-too-granular.-we-will-want-to-add-some-additional-columns-of-data-such-as-day-month-year-that-provide-additional-opportunities-to-aggregate-the-data.}}

\hypertarget{add-columns-that-list-the-date-year-month-day-and-day_of_week-of-each-ride}{%
\subparagraph{Add columns that list the date, year, month, day and
day\_of\_week of each
ride}\label{add-columns-that-list-the-date-year-month-day-and-day_of_week-of-each-ride}}

This will allow us to aggregate ride data for each month, day, or year
\ldots{} before completing these operations we could only aggregate at
the ride level

\begin{Shaded}
\begin{Highlighting}[]
\NormalTok{all\_trips}\SpecialCharTok{$}\NormalTok{date }\OtherTok{\textless{}{-}} \FunctionTok{as.Date}\NormalTok{(all\_trips}\SpecialCharTok{$}\NormalTok{started\_at)}
\NormalTok{all\_trips}\SpecialCharTok{$}\NormalTok{year }\OtherTok{\textless{}{-}} \FunctionTok{format}\NormalTok{(}\FunctionTok{as.Date}\NormalTok{(all\_trips}\SpecialCharTok{$}\NormalTok{date), }\StringTok{"\%Y"}\NormalTok{)}
\NormalTok{all\_trips}\SpecialCharTok{$}\NormalTok{month }\OtherTok{\textless{}{-}} \FunctionTok{format}\NormalTok{(}\FunctionTok{as.Date}\NormalTok{(all\_trips}\SpecialCharTok{$}\NormalTok{date), }\StringTok{"\%m"}\NormalTok{)}
\NormalTok{all\_trips}\SpecialCharTok{$}\NormalTok{day }\OtherTok{\textless{}{-}} \FunctionTok{format}\NormalTok{(}\FunctionTok{as.Date}\NormalTok{(all\_trips}\SpecialCharTok{$}\NormalTok{date), }\StringTok{"\%d"}\NormalTok{)}
\NormalTok{all\_trips}\SpecialCharTok{$}\NormalTok{day\_of\_week }\OtherTok{\textless{}{-}} \FunctionTok{format}\NormalTok{(}\FunctionTok{as.Date}\NormalTok{(all\_trips}\SpecialCharTok{$}\NormalTok{date), }\StringTok{"\%A"}\NormalTok{)}
\end{Highlighting}
\end{Shaded}

\hypertarget{we-will-add-a-calculated-field-for-length-of-ride-since-the-2020-q1-data-did-not-have-the-tripduration-column.-we-will-add-ride_length-to-the-entire-dataframe-for-consistency.}{%
\paragraph{(3) We will add a calculated field for length of ride since
the 2020 Q1 data did not have the ``tripduration'' column. We will add
``ride\_length'' to the entire dataframe for
consistency.}\label{we-will-add-a-calculated-field-for-length-of-ride-since-the-2020-q1-data-did-not-have-the-tripduration-column.-we-will-add-ride_length-to-the-entire-dataframe-for-consistency.}}

\hypertarget{add-a-ride_length-calculation-to-all_trips-in-seconds}{%
\subparagraph{Add a ``ride\_length'' calculation to all\_trips (in
seconds)}\label{add-a-ride_length-calculation-to-all_trips-in-seconds}}

\begin{Shaded}
\begin{Highlighting}[]
\NormalTok{all\_trips}\SpecialCharTok{$}\NormalTok{ride\_length }\OtherTok{\textless{}{-}} \FunctionTok{difftime}\NormalTok{(all\_trips}\SpecialCharTok{$}\NormalTok{ended\_at,all\_trips}\SpecialCharTok{$}\NormalTok{started\_at)}
\end{Highlighting}
\end{Shaded}

\hypertarget{inspect-the-structure-of-the-columns}{%
\subparagraph{Inspect the structure of the
columns}\label{inspect-the-structure-of-the-columns}}

\begin{Shaded}
\begin{Highlighting}[]
\FunctionTok{str}\NormalTok{(all\_trips)}
\end{Highlighting}
\end{Shaded}

\begin{verbatim}
## tibble [3,820,234 x 15] (S3: tbl_df/tbl/data.frame)
##  $ ride_id           : chr [1:3820234] "22178529" "22178530" "22178531" "22178532" ...
##  $ started_at        : POSIXct[1:3820234], format: "2019-04-01 00:02:00" "2019-04-01 00:03:00" ...
##  $ ended_at          : POSIXct[1:3820234], format: "2019-04-01 00:09:00" "2019-04-01 00:20:00" ...
##  $ rideable_type     : chr [1:3820234] "6251" "6226" "5649" "4151" ...
##  $ start_station_id  : num [1:3820234] 81 317 283 26 202 420 503 260 211 211 ...
##  $ start_station_name: chr [1:3820234] "Daley Center Plaza" "Wood St & Taylor St" "LaSalle St & Jackson Blvd" "McClurg Ct & Illinois St" ...
##  $ end_station_id    : num [1:3820234] 56 59 174 133 129 426 500 499 211 211 ...
##  $ end_station_name  : chr [1:3820234] "Desplaines St & Kinzie St" "Wabash Ave & Roosevelt Rd" "Canal St & Madison St" "Kingsbury St & Kinzie St" ...
##  $ member_casual     : chr [1:3820234] "member" "member" "member" "member" ...
##  $ date              : Date[1:3820234], format: "2019-04-01" "2019-04-01" ...
##  $ year              : chr [1:3820234] "2019" "2019" "2019" "2019" ...
##  $ month             : chr [1:3820234] "04" "04" "04" "04" ...
##  $ day               : chr [1:3820234] "01" "01" "01" "01" ...
##  $ day_of_week       : chr [1:3820234] "Monday" "Monday" "Monday" "Monday" ...
##  $ ride_length       : 'difftime' num [1:3820234] 420 1020 240 300 ...
##   ..- attr(*, "units")= chr "secs"
\end{verbatim}

\hypertarget{convert-ride_length-from-factor-to-numeric-so-we-can-run-calculations-on-the-data}{%
\subparagraph{Convert ``ride\_length'' from Factor to numeric so we can
run calculations on the
data}\label{convert-ride_length-from-factor-to-numeric-so-we-can-run-calculations-on-the-data}}

\begin{Shaded}
\begin{Highlighting}[]
\FunctionTok{is.factor}\NormalTok{(all\_trips}\SpecialCharTok{$}\NormalTok{ride\_length)}
\end{Highlighting}
\end{Shaded}

\begin{verbatim}
## [1] FALSE
\end{verbatim}

\begin{Shaded}
\begin{Highlighting}[]
\NormalTok{all\_trips}\SpecialCharTok{$}\NormalTok{ride\_length }\OtherTok{\textless{}{-}} \FunctionTok{as.numeric}\NormalTok{(}\FunctionTok{as.character}\NormalTok{(all\_trips}\SpecialCharTok{$}\NormalTok{ride\_length))}
\FunctionTok{is.numeric}\NormalTok{(all\_trips}\SpecialCharTok{$}\NormalTok{ride\_length)}
\end{Highlighting}
\end{Shaded}

\begin{verbatim}
## [1] TRUE
\end{verbatim}

\hypertarget{there-are-some-rides-where-tripduration-shows-up-as-negative-including-several-hundred-rides-where-divvy-took-bikes-out-of-circulation-for-quality-control-reasons.-we-will-want-to-delete-these-rides.}{%
\paragraph{(4) There are some rides where tripduration shows up as
negative, including several hundred rides where Divvy took bikes out of
circulation for Quality Control reasons. We will want to delete these
rides.}\label{there-are-some-rides-where-tripduration-shows-up-as-negative-including-several-hundred-rides-where-divvy-took-bikes-out-of-circulation-for-quality-control-reasons.-we-will-want-to-delete-these-rides.}}

\hypertarget{remove-bad-data}{%
\subparagraph{Remove ``bad'' data}\label{remove-bad-data}}

The dataframe includes a few hundred entries when bikes were taken out
of docks and checked for quality by Divvy or ride\_length was negative

\hypertarget{we-will-create-a-new-version-of-the-dataframe-v2-since-data-is-being-removed}{%
\subparagraph{We will create a new version of the dataframe (v2) since
data is being
removed}\label{we-will-create-a-new-version-of-the-dataframe-v2-since-data-is-being-removed}}

\begin{Shaded}
\begin{Highlighting}[]
\NormalTok{all\_trips\_v2 }\OtherTok{\textless{}{-}}\NormalTok{ all\_trips[}\SpecialCharTok{!}\NormalTok{(all\_trips}\SpecialCharTok{$}\NormalTok{start\_station\_name }\SpecialCharTok{==} \StringTok{"HQ QR"} \SpecialCharTok{|} 
\NormalTok{                              all\_trips}\SpecialCharTok{$}\NormalTok{ride\_length }\SpecialCharTok{\textless{}} \DecValTok{0}\NormalTok{),]}
\end{Highlighting}
\end{Shaded}

\hypertarget{step-4-conduct-descriptive-analysis}{%
\subsubsection{STEP 4: CONDUCT DESCRIPTIVE
ANALYSIS}\label{step-4-conduct-descriptive-analysis}}

\hypertarget{descriptive-analysis-on-ride_length-all-figures-in-seconds}{%
\paragraph{Descriptive analysis on ride\_length (all figures in
seconds)}\label{descriptive-analysis-on-ride_length-all-figures-in-seconds}}

\begin{Shaded}
\begin{Highlighting}[]
\FunctionTok{mean}\NormalTok{(all\_trips\_v2}\SpecialCharTok{$}\NormalTok{ride\_length) }\CommentTok{\#straight average (total ride length / rides)}
\end{Highlighting}
\end{Shaded}

\begin{verbatim}
## [1] 1476.792
\end{verbatim}

\begin{Shaded}
\begin{Highlighting}[]
\FunctionTok{median}\NormalTok{(all\_trips\_v2}\SpecialCharTok{$}\NormalTok{ride\_length) }\CommentTok{\#midpoint number in the ascending array of ride lengths}
\end{Highlighting}
\end{Shaded}

\begin{verbatim}
## [1] 716
\end{verbatim}

\begin{Shaded}
\begin{Highlighting}[]
\FunctionTok{max}\NormalTok{(all\_trips\_v2}\SpecialCharTok{$}\NormalTok{ride\_length) }\CommentTok{\#longest ride}
\end{Highlighting}
\end{Shaded}

\begin{verbatim}
## [1] 9387024
\end{verbatim}

\begin{Shaded}
\begin{Highlighting}[]
\FunctionTok{min}\NormalTok{(all\_trips\_v2}\SpecialCharTok{$}\NormalTok{ride\_length) }\CommentTok{\#shortest ride}
\end{Highlighting}
\end{Shaded}

\begin{verbatim}
## [1] 1
\end{verbatim}

\hypertarget{condense-the-four-lines-above-to-one-line-using-summary-on-the-specific-attribute}{%
\paragraph{Condense the four lines above to one line using summary() on
the specific
attribute}\label{condense-the-four-lines-above-to-one-line-using-summary-on-the-specific-attribute}}

\begin{Shaded}
\begin{Highlighting}[]
\FunctionTok{summary}\NormalTok{(all\_trips\_v2}\SpecialCharTok{$}\NormalTok{ride\_length)}
\end{Highlighting}
\end{Shaded}

\begin{verbatim}
##    Min. 1st Qu.  Median    Mean 3rd Qu.    Max. 
##       1     418     716    1477    1279 9387024
\end{verbatim}

\hypertarget{compare-members-and-casual-users}{%
\paragraph{Compare members and casual
users}\label{compare-members-and-casual-users}}

\begin{Shaded}
\begin{Highlighting}[]
\FunctionTok{aggregate}\NormalTok{(all\_trips\_v2}\SpecialCharTok{$}\NormalTok{ride\_length }\SpecialCharTok{\textasciitilde{}}\NormalTok{ all\_trips\_v2}\SpecialCharTok{$}\NormalTok{member\_casual, }\AttributeTok{FUN =}\NormalTok{ mean)}
\end{Highlighting}
\end{Shaded}

\begin{verbatim}
##   all_trips_v2$member_casual all_trips_v2$ride_length
## 1                     casual                3575.3095
## 2                     member                 849.2708
\end{verbatim}

\begin{Shaded}
\begin{Highlighting}[]
\FunctionTok{aggregate}\NormalTok{(all\_trips\_v2}\SpecialCharTok{$}\NormalTok{ride\_length }\SpecialCharTok{\textasciitilde{}}\NormalTok{ all\_trips\_v2}\SpecialCharTok{$}\NormalTok{member\_casual, }\AttributeTok{FUN =}\NormalTok{ median)}
\end{Highlighting}
\end{Shaded}

\begin{verbatim}
##   all_trips_v2$member_casual all_trips_v2$ride_length
## 1                     casual                     1548
## 2                     member                      596
\end{verbatim}

\begin{Shaded}
\begin{Highlighting}[]
\FunctionTok{aggregate}\NormalTok{(all\_trips\_v2}\SpecialCharTok{$}\NormalTok{ride\_length }\SpecialCharTok{\textasciitilde{}}\NormalTok{ all\_trips\_v2}\SpecialCharTok{$}\NormalTok{member\_casual, }\AttributeTok{FUN =}\NormalTok{ max)}
\end{Highlighting}
\end{Shaded}

\begin{verbatim}
##   all_trips_v2$member_casual all_trips_v2$ride_length
## 1                     casual                  9387024
## 2                     member                  9056634
\end{verbatim}

\begin{Shaded}
\begin{Highlighting}[]
\FunctionTok{aggregate}\NormalTok{(all\_trips\_v2}\SpecialCharTok{$}\NormalTok{ride\_length }\SpecialCharTok{\textasciitilde{}}\NormalTok{ all\_trips\_v2}\SpecialCharTok{$}\NormalTok{member\_casual, }\AttributeTok{FUN =}\NormalTok{ min)}
\end{Highlighting}
\end{Shaded}

\begin{verbatim}
##   all_trips_v2$member_casual all_trips_v2$ride_length
## 1                     casual                        2
## 2                     member                        1
\end{verbatim}

\hypertarget{see-the-average-ride-time-by-each-day-for-members-vs-casual-users}{%
\paragraph{See the average ride time by each day for members vs casual
users}\label{see-the-average-ride-time-by-each-day-for-members-vs-casual-users}}

\begin{Shaded}
\begin{Highlighting}[]
\FunctionTok{aggregate}\NormalTok{(all\_trips\_v2}\SpecialCharTok{$}\NormalTok{ride\_length }\SpecialCharTok{\textasciitilde{}}\NormalTok{ all\_trips\_v2}\SpecialCharTok{$}\NormalTok{member\_casual }\SpecialCharTok{+} 
\NormalTok{            all\_trips\_v2}\SpecialCharTok{$}\NormalTok{day\_of\_week, }\AttributeTok{FUN =}\NormalTok{ mean)}
\end{Highlighting}
\end{Shaded}

\begin{verbatim}
##    all_trips_v2$member_casual all_trips_v2$day_of_week all_trips_v2$ride_length
## 1                      casual                   Friday                3822.8518
## 2                      member                   Friday                 824.4114
## 3                      casual                   Monday                3372.3062
## 4                      member                   Monday                 842.5778
## 5                      casual                 Saturday                3362.5847
## 6                      member                 Saturday                 968.0975
## 7                      casual                   Sunday                3607.8092
## 8                      member                   Sunday                 919.0421
## 9                      casual                 Thursday                3705.9946
## 10                     member                 Thursday                 822.8516
## 11                     casual                  Tuesday                3596.3748
## 12                     member                  Tuesday                 826.1573
## 13                     casual                Wednesday                3718.7743
## 14                     member                Wednesday                 823.9830
\end{verbatim}

\hypertarget{notice-that-the-days-of-the-week-are-out-of-order.-lets-fix-that.}{%
\paragraph{Notice that the days of the week are out of order. Let's fix
that.}\label{notice-that-the-days-of-the-week-are-out-of-order.-lets-fix-that.}}

\begin{Shaded}
\begin{Highlighting}[]
\NormalTok{all\_trips\_v2}\SpecialCharTok{$}\NormalTok{day\_of\_week }\OtherTok{\textless{}{-}} \FunctionTok{ordered}\NormalTok{(all\_trips\_v2}\SpecialCharTok{$}\NormalTok{day\_of\_week, }
                                    \AttributeTok{levels =} \FunctionTok{c}\NormalTok{(}\StringTok{"Sunday"}\NormalTok{, }\StringTok{"Monday"}\NormalTok{, }
                                             \StringTok{"Tuesday"}\NormalTok{, }\StringTok{"Wednesday"}\NormalTok{, }
                                             \StringTok{"Thursday"}\NormalTok{, }\StringTok{"Friday"}\NormalTok{, }
                                             \StringTok{"Saturday"}\NormalTok{))}
\end{Highlighting}
\end{Shaded}

\hypertarget{now-lets-run-the-average-ride-time-by-each-day-for-members-vs-casual-users}{%
\paragraph{Now, let's run the average ride time by each day for members
vs casual
users}\label{now-lets-run-the-average-ride-time-by-each-day-for-members-vs-casual-users}}

\begin{Shaded}
\begin{Highlighting}[]
\FunctionTok{aggregate}\NormalTok{(all\_trips\_v2}\SpecialCharTok{$}\NormalTok{ride\_length }\SpecialCharTok{\textasciitilde{}}\NormalTok{ all\_trips\_v2}\SpecialCharTok{$}\NormalTok{member\_casual }\SpecialCharTok{+} 
\NormalTok{            all\_trips\_v2}\SpecialCharTok{$}\NormalTok{day\_of\_week, }\AttributeTok{FUN =}\NormalTok{ mean)}
\end{Highlighting}
\end{Shaded}

\begin{verbatim}
##    all_trips_v2$member_casual all_trips_v2$day_of_week all_trips_v2$ride_length
## 1                      casual                   Sunday                3607.8092
## 2                      member                   Sunday                 919.0421
## 3                      casual                   Monday                3372.3062
## 4                      member                   Monday                 842.5778
## 5                      casual                  Tuesday                3596.3748
## 6                      member                  Tuesday                 826.1573
## 7                      casual                Wednesday                3718.7743
## 8                      member                Wednesday                 823.9830
## 9                      casual                 Thursday                3705.9946
## 10                     member                 Thursday                 822.8516
## 11                     casual                   Friday                3822.8518
## 12                     member                   Friday                 824.4114
## 13                     casual                 Saturday                3362.5847
## 14                     member                 Saturday                 968.0975
\end{verbatim}

\hypertarget{analyze-ridership-data-by-type-and-weekday}{%
\paragraph{analyze ridership data by type and
weekday}\label{analyze-ridership-data-by-type-and-weekday}}

\begin{Shaded}
\begin{Highlighting}[]
\NormalTok{all\_trips\_v2 }\SpecialCharTok{\%\textgreater{}\%} 
  \FunctionTok{mutate}\NormalTok{(}\AttributeTok{weekday =} \FunctionTok{wday}\NormalTok{(started\_at, }\AttributeTok{label =} \ConstantTok{TRUE}\NormalTok{)) }\SpecialCharTok{\%\textgreater{}\%}  \CommentTok{\#creates weekday field using wday()}
  \FunctionTok{group\_by}\NormalTok{(member\_casual, weekday) }\SpecialCharTok{\%\textgreater{}\%}  \CommentTok{\#groups by usertype and weekday}
  \FunctionTok{summarise}\NormalTok{(}\AttributeTok{number\_of\_rides =} \FunctionTok{n}\NormalTok{(),          }\CommentTok{\#calculates the number of rides and average duration }
  \AttributeTok{average\_duration =} \FunctionTok{mean}\NormalTok{(ride\_length)) }\SpecialCharTok{\%\textgreater{}\%}         \CommentTok{\# calculates the average duration}
  \FunctionTok{arrange}\NormalTok{(member\_casual, weekday)                               }\CommentTok{\# sorts}
\end{Highlighting}
\end{Shaded}

\begin{verbatim}
## `summarise()` has grouped output by 'member_casual'. You can override using the `.groups` argument.
\end{verbatim}

\begin{verbatim}
## # A tibble: 14 x 4
## # Groups:   member_casual [2]
##    member_casual weekday number_of_rides average_duration
##    <chr>         <ord>             <int>            <dbl>
##  1 casual        Sun              174768            3608.
##  2 casual        Mon              103296            3372.
##  3 casual        Tue               90510            3596.
##  4 casual        Wed               92457            3719.
##  5 casual        Thu              101142            3706.
##  6 casual        Fri              117753            3823.
##  7 casual        Sat              198604            3363.
##  8 member        Sun              260608             919.
##  9 member        Mon              472196             843.
## 10 member        Tue              508445             826.
## 11 member        Wed              500329             824.
## 12 member        Thu              479341             823.
## 13 member        Fri              439526             824.
## 14 member        Sat              277479             968.
\end{verbatim}

\hypertarget{lets-visualize-the-number-of-rides-by-rider-type}{%
\paragraph{Let's visualize the number of rides by rider
type}\label{lets-visualize-the-number-of-rides-by-rider-type}}

\begin{Shaded}
\begin{Highlighting}[]
\NormalTok{all\_trips\_v2 }\SpecialCharTok{\%\textgreater{}\%} 
  \FunctionTok{mutate}\NormalTok{(}\AttributeTok{weekday =} \FunctionTok{wday}\NormalTok{(started\_at, }\AttributeTok{label =} \ConstantTok{TRUE}\NormalTok{)) }\SpecialCharTok{\%\textgreater{}\%} 
  \FunctionTok{group\_by}\NormalTok{(member\_casual, weekday) }\SpecialCharTok{\%\textgreater{}\%} 
  \FunctionTok{summarise}\NormalTok{(}\AttributeTok{number\_of\_rides =} \FunctionTok{n}\NormalTok{(), }
            \AttributeTok{average\_duration =} \FunctionTok{mean}\NormalTok{(ride\_length)) }\SpecialCharTok{\%\textgreater{}\%} 
  \FunctionTok{arrange}\NormalTok{(member\_casual, weekday)  }\SpecialCharTok{\%\textgreater{}\%} 
  \FunctionTok{ggplot}\NormalTok{(}\FunctionTok{aes}\NormalTok{(}\AttributeTok{x =}\NormalTok{ weekday, }\AttributeTok{y =}\NormalTok{ number\_of\_rides, }
             \AttributeTok{fill =}\NormalTok{ member\_casual)) }\SpecialCharTok{+}
  \FunctionTok{geom\_col}\NormalTok{(}\AttributeTok{position =} \StringTok{"dodge"}\NormalTok{)}
\end{Highlighting}
\end{Shaded}

\begin{verbatim}
## `summarise()` has grouped output by 'member_casual'. You can override using the `.groups` argument.
\end{verbatim}

\includegraphics{01_cyclistic_files/figure-latex/visualize the number of rides by rider type-1.pdf}

\hypertarget{lets-create-a-visualization-for-average-duration}{%
\paragraph{Let's create a visualization for average
duration}\label{lets-create-a-visualization-for-average-duration}}

\begin{Shaded}
\begin{Highlighting}[]
\NormalTok{all\_trips\_v2 }\SpecialCharTok{\%\textgreater{}\%} 
  \FunctionTok{mutate}\NormalTok{(}\AttributeTok{weekday =} \FunctionTok{wday}\NormalTok{(started\_at, }\AttributeTok{label =} \ConstantTok{TRUE}\NormalTok{)) }\SpecialCharTok{\%\textgreater{}\%} 
  \FunctionTok{group\_by}\NormalTok{(member\_casual, weekday) }\SpecialCharTok{\%\textgreater{}\%} 
  \FunctionTok{summarise}\NormalTok{(}\AttributeTok{number\_of\_rides =} \FunctionTok{n}\NormalTok{(), }
            \AttributeTok{average\_duration =} \FunctionTok{mean}\NormalTok{(ride\_length)) }\SpecialCharTok{\%\textgreater{}\%}
  \FunctionTok{arrange}\NormalTok{(member\_casual, weekday) }\SpecialCharTok{\%\textgreater{}\%} 
  \FunctionTok{ggplot}\NormalTok{(}\FunctionTok{aes}\NormalTok{(}\AttributeTok{x =}\NormalTok{ weekday, }\AttributeTok{y =}\NormalTok{ average\_duration, }
             \AttributeTok{fill =}\NormalTok{ member\_casual)) }\SpecialCharTok{+}
  \FunctionTok{geom\_col}\NormalTok{(}\AttributeTok{position =} \StringTok{"dodge"}\NormalTok{)}
\end{Highlighting}
\end{Shaded}

\begin{verbatim}
## `summarise()` has grouped output by 'member_casual'. You can override using the `.groups` argument.
\end{verbatim}

\includegraphics{01_cyclistic_files/figure-latex/create a visualization for average duration-1.pdf}

\hypertarget{step-5-export-summary-file-for-further-analysis}{%
\subsubsection{STEP 5: EXPORT SUMMARY FILE FOR FURTHER
ANALYSIS}\label{step-5-export-summary-file-for-further-analysis}}

\hypertarget{create-a-csv-file-that-we-will-visualize-in-excel-tableau-or-my-presentation-software}{%
\paragraph{Create a csv file that we will visualize in Excel, Tableau,
or my presentation
software}\label{create-a-csv-file-that-we-will-visualize-in-excel-tableau-or-my-presentation-software}}

\begin{Shaded}
\begin{Highlighting}[]
\NormalTok{counts }\OtherTok{\textless{}{-}} \FunctionTok{aggregate}\NormalTok{(all\_trips\_v2}\SpecialCharTok{$}\NormalTok{ride\_length }\SpecialCharTok{\textasciitilde{}}\NormalTok{ all\_trips\_v2}\SpecialCharTok{$}\NormalTok{member\_casual }\SpecialCharTok{+} 
\NormalTok{                      all\_trips\_v2}\SpecialCharTok{$}\NormalTok{day\_of\_week, }\AttributeTok{FUN =}\NormalTok{ mean)}
\FunctionTok{write.csv}\NormalTok{(counts, }\AttributeTok{file =}  \StringTok{\textquotesingle{}C:/Users/Action/Documents/RStudio/repos/01\_\_cyclistic/avg\_ride\_length.csv\textquotesingle{}}\NormalTok{)    }
\end{Highlighting}
\end{Shaded}


\end{document}
